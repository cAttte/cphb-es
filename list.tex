\begin{thebibliography}{9}

\bibitem{aho83}
  A. V. Aho, J. E. Hopcroft and J. Ullman.
  \emph{Data Structures and Algorithms},
  Addison-Wesley, 1983.

\bibitem{ahu91}
  R. K. Ahuja and J. B. Orlin.
  Distance directed augmenting path algorithms for maximum flow and parametric maximum flow problems.
  \emph{Naval Research Logistics}, 38(3):413--430, 1991.

\bibitem{and79}
  A. M. Andrew.
  Another efficient algorithm for convex hulls in two dimensions.
  \emph{Information Processing Letters}, 9(5):216--219, 1979.

\bibitem{asp79}
  B. Aspvall, M. F. Plass and R. E. Tarjan.
  A linear-time algorithm for testing the truth of certain quantified boolean formulas.
  \emph{Information Processing Letters}, 8(3):121--123, 1979.

\bibitem{bel58}
  R. Bellman.
  On a routing problem.
  \emph{Quarterly of Applied Mathematics}, 16(1):87--90, 1958.

\bibitem{bec07}
  M. Beck, E. Pine, W. Tarrat and K. Y. Jensen.
  New integer representations as the sum of three cubes.
  \emph{Mathematics of Computation}, 76(259):1683--1690, 2007.

\bibitem{ben00}
  M. A. Bender and M. Farach-Colton.
  The LCA problem revisited. In
  \emph{Latin American Symposium on Theoretical Informatics}, 88--94, 2000.

\bibitem{ben86}
  J. Bentley.
  \emph{Programming Pearls}.
  Addison-Wesley, 1999 (2nd edition).

\bibitem{ben80}
  J. Bentley and D. Wood.
  An optimal worst case algorithm for reporting intersections of rectangles.
  \emph{IEEE Transactions on Computers}, C-29(7):571--577, 1980.

\bibitem{bou01}
  C. L. Bouton.
  Nim, a game with a complete mathematical theory.
  \emph{Annals of Mathematics}, 3(1/4):35--39, 1901.

% \bibitem{bur97}
%   W. Burnside.
%   \emph{Theory of Groups of Finite Order},
%   Cambridge University Press, 1897.

\bibitem{coci}
  Croatian Open Competition in Informatics, \url{http://hsin.hr/coci/}

\bibitem{cod15}
  Codeforces: On ''Mo's algorithm'',
  \url{http://codeforces.com/blog/entry/20032}

\bibitem{cor09}
  T. H. Cormen, C. E. Leiserson, R. L. Rivest and C. Stein.
  \emph{Introduction to Algorithms}, MIT Press, 2009 (3rd edition).

\bibitem{dij59}
  E. W. Dijkstra.
  A note on two problems in connexion with graphs.
  \emph{Numerische Mathematik}, 1(1):269--271, 1959.

\bibitem{dik12}
  K. Diks et al.
  \emph{Looking for a Challenge? The Ultimate Problem Set from
  the University of Warsaw Programming Competitions}, University of Warsaw, 2012.

% \bibitem{dil50}
%   R. P. Dilworth.
%   A decomposition theorem for partially ordered sets.
%   \emph{Annals of Mathematics}, 51(1):161--166, 1950.

% \bibitem{dir52}
%   G. A. Dirac.
%   Some theorems on abstract graphs.
%   \emph{Proceedings of the London Mathematical Society}, 3(1):69--81, 1952.

\bibitem{dim15}
  M. Dima and R. Ceterchi.
  Efficient range minimum queries using binary indexed trees.
  \emph{Olympiad in Informatics}, 9(1):39--44, 2015.

\bibitem{edm65}
  J. Edmonds.
  Paths, trees, and flowers.
  \emph{Canadian Journal of Mathematics}, 17(3):449--467, 1965.

\bibitem{edm72}
  J. Edmonds and R. M. Karp.
  Theoretical improvements in algorithmic efficiency for network flow problems.
  \emph{Journal of the ACM}, 19(2):248--264, 1972.

\bibitem{eve75}
  S. Even, A. Itai and A. Shamir.
  On the complexity of time table and multi-commodity flow problems.
  \emph{16th Annual Symposium on Foundations of Computer Science}, 184--193, 1975.

\bibitem{fan94}
  D. Fanding.
  A faster algorithm for shortest-path -- SPFA.
  \emph{Journal of Southwest Jiaotong University}, 2, 1994.

\bibitem{fen94}
  P. M. Fenwick.
  A new data structure for cumulative frequency tables.
  \emph{Software: Practice and Experience}, 24(3):327--336, 1994.

\bibitem{fis06}
  J. Fischer and V. Heun.
  Theoretical and practical improvements on the RMQ-problem, with applications to LCA and LCE.
  In \emph{Annual Symposium on Combinatorial Pattern Matching}, 36--48, 2006.

\bibitem{flo62}
  R. W. Floyd
  Algorithm 97: shortest path.
  \emph{Communications of the ACM}, 5(6):345, 1962.

\bibitem{for56a}
  L. R. Ford.
  Network flow theory.
  RAND Corporation, Santa Monica, California, 1956.

\bibitem{for56}
  L. R. Ford and D. R. Fulkerson.
  Maximal flow through a network.
  \emph{Canadian Journal of Mathematics}, 8(3):399--404, 1956.

\bibitem{fre77}
  R. Freivalds.
  Probabilistic machines can use less running time.
  In \emph{IFIP congress}, 839--842, 1977.

\bibitem{gal14}
  F. Le Gall.
  Powers of tensors and fast matrix multiplication.
  In \emph{Proceedings of the 39th International Symposium on Symbolic and Algebraic Computation},
  296--303, 2014.

\bibitem{gar79}
  M. R. Garey and D. S. Johnson.
  \emph{Computers and Intractability:
  A Guide to the Theory of NP-Completeness},
  W. H. Freeman and Company, 1979.

\bibitem{goo17}
  Google Code Jam Statistics (2017),
  \url{https://www.go-hero.net/jam/17}

\bibitem{gro14}
  A. Grønlund and S. Pettie.
  Threesomes, degenerates, and love triangles.
  In \emph{Proceedings of the 55th Annual Symposium on Foundations of Computer Science},
  621--630, 2014.

\bibitem{gru39}
  P. M. Grundy.
  Mathematics and games.
  \emph{Eureka}, 2(5):6--8, 1939.

\bibitem{gus97}
  D. Gusfield.
  \emph{Algorithms on Strings, Trees and Sequences:
  Computer Science and Computational Biology},
  Cambridge University Press, 1997.

% \bibitem{hal35}
%   P. Hall.
%   On representatives of subsets.
%   \emph{Journal London Mathematical Society} 10(1):26--30, 1935.

\bibitem{hal21}
  S. Halim, F. Halim y S. Effendy.
  \emph{Programación competitiva 4: Manual para concursantes del ICPC y la IOI}, OJ Books, 2021.

\bibitem{hel62}
  M. Held and R. M. Karp.
  A dynamic programming approach to sequencing problems.
  \emph{Journal of the Society for Industrial and Applied Mathematics}, 10(1):196--210, 1962.

\bibitem{hie73}
  C. Hierholzer and C. Wiener.
  Über die Möglichkeit, einen Linienzug ohne Wiederholung und ohne Unterbrechung zu umfahren.
  \emph{Mathematische Annalen}, 6(1), 30--32, 1873.

\bibitem{hoa61a}
  C. A. R. Hoare.
  Algorithm 64: Quicksort.
  \emph{Communications of the ACM}, 4(7):321, 1961.

\bibitem{hoa61b}
  C. A. R. Hoare.
  Algorithm 65: Find.
  \emph{Communications of the ACM}, 4(7):321--322, 1961.

\bibitem{hop71}
  J. E. Hopcroft and J. D. Ullman.
  A linear list merging algorithm.
  Technical report, Cornell University, 1971.

\bibitem{hor74}
  E. Horowitz and S. Sahni.
  Computing partitions with applications to the knapsack problem.
  \emph{Journal of the ACM}, 21(2):277--292, 1974.

\bibitem{huf52}
  D. A. Huffman.
  A method for the construction of minimum-redundancy codes.
  \emph{Proceedings of the IRE}, 40(9):1098--1101, 1952.

\bibitem{iois}
  The International Olympiad in Informatics Syllabus,
  \url{https://people.ksp.sk/~misof/ioi-syllabus/}

\bibitem{kar87}
  R. M. Karp and M. O. Rabin.
  Efficient randomized pattern-matching algorithms.
  \emph{IBM Journal of Research and Development}, 31(2):249--260, 1987.

\bibitem{kas61}
  P. W. Kasteleyn.
  The statistics of dimers on a lattice: I. The number of dimer arrangements on a quadratic lattice.
  \emph{Physica}, 27(12):1209--1225, 1961.

\bibitem{ken06}
  C. Kent, G. M. Landau and M. Ziv-Ukelson.
  On the complexity of sparse exon assembly.
  \emph{Journal of Computational Biology}, 13(5):1013--1027, 2006.


\bibitem{kle05}
  J. Kleinberg and É. Tardos.
  \emph{Algorithm Design}, Pearson, 2005.

\bibitem{knu982}
  D. E. Knuth.
  \emph{The Art of Computer Programming. Volume 2: Seminumerical Algorithms}, Addison–Wesley, 1998 (3rd edition).

\bibitem{knu983}
  D. E. Knuth.
  \emph{The Art of Computer Programming. Volume 3: Sorting and Searching}, Addison–Wesley, 1998 (2nd edition).

% \bibitem{kon31}
%   D. Kőnig.
%   Gráfok és mátrixok.
%   \emph{Matematikai és Fizikai Lapok}, 38(1):116--119, 1931.

\bibitem{kru56}
  J. B. Kruskal.
  On the shortest spanning subtree of a graph and the traveling salesman problem.
  \emph{Proceedings of the American Mathematical Society}, 7(1):48--50, 1956.

\bibitem{lev66}
  V. I. Levenshtein.
  Binary codes capable of correcting deletions, insertions, and reversals.
  \emph{Soviet physics doklady}, 10(8):707--710, 1966.

\bibitem{mai84}
  M. G. Main and R. J. Lorentz.
  An $O(n \log n)$ algorithm for finding all repetitions in a string.
  \emph{Journal of Algorithms}, 5(3):422--432, 1984.

% \bibitem{ore60}
%   Ø. Ore.
%   Note on Hamilton circuits.
%   \emph{The American Mathematical Monthly}, 67(1):55, 1960.

\bibitem{pac13}
  J. Pachocki and J. Radoszewski.
  Where to use and how not to use polynomial string hashing.
  \emph{Olympiads in Informatics}, 7(1):90--100, 2013.

\bibitem{par97}
  I. Parberry.
  An efficient algorithm for the Knight's tour problem.
  \emph{Discrete Applied Mathematics}, 73(3):251--260, 1997.

% \bibitem{pic99}
%   G. Pick.
%   Geometrisches zur Zahlenlehre.
%   \emph{Sitzungsberichte des deutschen naturwissenschaftlich-medicinischen Vereines
%   für Böhmen "Lotos" in Prag. (Neue Folge)}, 19:311--319, 1899.

\bibitem{pea05}
  D. Pearson.
  A polynomial-time algorithm for the change-making problem.
  \emph{Operations Research Letters}, 33(3):231--234, 2005.

\bibitem{pri57}
  R. C. Prim.
  Shortest connection networks and some generalizations.
  \emph{Bell System Technical Journal}, 36(6):1389--1401, 1957.

% \bibitem{pru18}
%   H. Prüfer.
%   Neuer Beweis eines Satzes über Permutationen.
%   \emph{Arch. Math. Phys}, 27:742--744, 1918.

\bibitem{q27}
  27-Queens Puzzle: Massively Parallel Enumeration and Solution Counting.
  \url{https://github.com/preusser/q27}

\bibitem{sha75}
  M. I. Shamos and D. Hoey.
  Closest-point problems.
  In \emph{Proceedings of the 16th Annual Symposium on Foundations of Computer Science}, 151--162, 1975.

\bibitem{sha81}
  M. Sharir.
  A strong-connectivity algorithm and its applications in data flow analysis.
  \emph{Computers \& Mathematics with Applications}, 7(1):67--72, 1981.

\bibitem{ski08}
  S. S. Skiena.
  \emph{The Algorithm Design Manual}, Springer, 2008 (2nd edition).

\bibitem{ski20}
  S. S. Skiena y M. A. Revilla.
  \emph{Desafíos de programación: El manual de entrenamiento para concursos de programación},
  OJ Books, 2020.

\bibitem{main}
  SZKOpuł, \texttt{https://szkopul.edu.pl/}

\bibitem{spr35}
  R. Sprague.
  Über mathematische Kampfspiele.
  \emph{Tohoku Mathematical Journal}, 41:438--444, 1935.

\bibitem{sta06}
  P. Stańczyk.
  \emph{Algorytmika praktyczna w konkursach Informatycznych},
  MSc thesis, University of Warsaw, 2006.

\bibitem{str69}
  V. Strassen.
  Gaussian elimination is not optimal.
  \emph{Numerische Mathematik}, 13(4):354--356, 1969.

\bibitem{tar75}
  R. E. Tarjan.
  Efficiency of a good but not linear set union algorithm.
  \emph{Journal of the ACM}, 22(2):215--225, 1975.

\bibitem{tar79}
  R. E. Tarjan.
  Applications of path compression on balanced trees.
  \emph{Journal of the ACM}, 26(4):690--715, 1979.

\bibitem{tar84}
  R. E. Tarjan and U. Vishkin.
  Finding biconnected componemts and computing tree functions in logarithmic parallel time.
  In \emph{Proceedings of the 25th Annual Symposium on Foundations of Computer Science}, 12--20, 1984.

\bibitem{tem61}
  H. N. V. Temperley and M. E. Fisher.
  Dimer problem in statistical mechanics -- an exact result.
  \emph{Philosophical Magazine}, 6(68):1061--1063, 1961.

\bibitem{usaco}
  USA Computing Olympiad, \url{http://www.usaco.org/}

\bibitem{war23}
  H. C. von Warnsdorf.
  \emph{Des Rösselsprunges einfachste und allgemeinste Lösung}.
  Schmalkalden, 1823.

\bibitem{war62}
  S. Warshall.
  A theorem on boolean matrices.
  \emph{Journal of the ACM}, 9(1):11--12, 1962.

% \bibitem{zec72}
%   E. Zeckendorf.
%   Représentation des nombres naturels par une somme de nombres de Fibonacci ou de nombres de Lucas.
%   \emph{Bull. Soc. Roy. Sci. Liege}, 41:179--182, 1972.

\end{thebibliography}
